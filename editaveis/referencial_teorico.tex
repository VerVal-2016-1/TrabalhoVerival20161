\chapter{Referencial Teórico}
Para realizar o levantamento do referencial teórico foi utilizado a técnica de revisão de literatura.
Segundo \citeonline{revisao_de_literatura}, uma revisão de literatura comporta uma parte muito importante no processo de investigação, 
uma vez estabelecida por localizar, analisar, sintetizar e interpretar a investigação prévia referente a uma área de estudo. 
Além disso, uma boa revisão de literatura auxilia no entendimento de um problema e desenvolvimento dos conhecimentos, 
conforme \apudonline{livro_sistematizacao_conhecimento}{revisao_de_literatura}, “cada investigador analisa minuciosamente os trabalhos dos investigadores 
que o precederam e, só então, compreendido o testemunho que lhe foi confiado, parte equipado para a sua própria aventura”, 
ou seja, ao se iniciar o processo de revisão de literatura raramente não existirão assuntos abordados no passado, sobre o 
mesmo tema ou similares, que auxiliarão como fonte de conhecimento para a atual pesquisa.

Novamente, segundo \citeonline{revisao_de_literatura}, os propósitos da revisão de literatura num estudo de investigação são:

\begin{itemize}
 
	\item \textbf{Delimitar o problema de investigação:} formular uma definição concreta sobre o que será investigado visando não comprometer todo o trabalho com problemas mal delimitados.
	\item \textbf{Procurar novas linhas de investigação:} consiste em entender o que já foi realizado sobre um determinado problema e buscar pelas suas áreas que ainda não foram ou foram pouco exploradas.
	\item \textbf{Evitar abordagens infrutíferas:} buscar que as linhas de investigação definidas sejam proveitosas e possuam resultados significativos.
	\item \textbf{Ganhar perspectivas metodológicas:} não rever apenas os resultados do estudo, mas sim, realizar uma leitura geral sobre todos os tópicos abordados.
	\item \textbf{Identificar recomendações para investigações futuras:} após finalizado o estudo, identificar novas questões e sugestões para investigações futuras.

\end{itemize}

O referencial teórico deste trabalho contemplará os conceitos de:

\begin{itemize}

	\item Testes unitarios
	\item Testes de integração
	\item \textit{Framework} de testes
	
\end{itemize}

