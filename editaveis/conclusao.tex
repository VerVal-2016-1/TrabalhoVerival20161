\chapter{Conclusão}

  Com a aplicação do questionário a desenvolvedores PHP, foi comprovado que o \textit{Ignitest} traz benefícios reais para a criação de testes.
  O Ignitest diminuiu a quantidade de código de testes, o que motiva o desenvolvedor a testar mais, além de reduzir expressívelmente o tempo de criação dos testes.
  Outro fator vantajoso do \textit{Ignitest}, é que por sua estrutura de comandos ser simples, os erros inseridos pelo usuário são reduzidos, conforme visto nas respostas dos questionários.
  
  Sem a utilização do \textit{Ignitest}, algo que poderia ocorrer com certa facilidade era a duplicação de código, principalmente alguém sem experiência com testes em PHP, já com o uso do \textit{Ignitest} essa preocupação é reduzida.
  
  No que diz respeito a instalação e configuração do \textit{framework}, os avaliadores não relataram problemas, demonstrando que a documentação de apoio é suficiente para as atividades mencionadas.
  
  Em relação a metodologia de pesquisa utilizada, é possível notar que por meio da pesquisa-ação os resultados são obtidos de forma mais rápida. Além da redução de tempo, existe também a redução de dúvidas acerca dos problemas, tendo em vista que o pesquisador está agindo dentro do caso estudado.
  
  É importante ressaltar que a versão do \textit{Ignitest} aqui apresentada é inicial, sendo assim reduzida em funcionalidades. Porém, ao mesmo tempo, por ser \textit{open-source}, abre oportunidades para sua evolução, por meio da implementação de novas funcionalidades ou pesquisas futuras relativas ao seu uso.