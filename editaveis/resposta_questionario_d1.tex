\chapter{Respostas da entrevista - Desenvolvedor 1}
    
    \begin{itemize}
        \item Documentação
        
            Com base na documentação presente no repositório do \textit{Ignitest}\footnote{https://github.com/VerVal-2016-1/Framework}, responda:
            \begin{itemize}
                \item \textbf{Com as informações presentes na seção "Instalação", foi possível instalar o \textit{Ignitest} sem dificuldades?}
		    
		    \subitem \textit{O processo de instalação do Framework é bastante simples e, embora simples, está bem documentado no 
			      repositório do Framework}.
		    
                \item \textbf{Com as informações presentes na seção "Instruções", foi possível entender o funcionamento do \textit{Ignitest}?}
		     
		     \subitem \textit{As funcionalidades do Framework estão bem descritas no repositório, com suas respectivas configurações.
			      Foi bastante simples entender como usar o Framework}.
		     
            \end{itemize}
            
        \item Resultados
            \begin{itemize}
                \item Tempo
                    \begin{enumerate}
                        \item \textbf{Qual o tempo gasto para testar uma determinada classe, usando o método padrão?}
			  
			  \subitem \textit{Foram gastos 45 minutos utilizando o método padrão para realizar a implementação de 15 casos de testes.}
			  
                        \item \textbf{Qual o tempo gasto para testar a mesma classe com o auxílio do \textit{Ignitest}?}
			  
			  \subitem \textit{Para os mesmos 15 casos de testes, foram gastos 16 minutos utilizando o Ignitest.}
                        
                    \end{enumerate}
                \item Dificuldades
                    \begin{enumerate}
                        \item \textbf{Existem dificuldades em testar usando o método padrão? Se sim, quais?}
                        
			  \subitem \textit{Sim. No método padrão é necessário criar uma controller para representar a classe de teste,
				    criar uma rota para a controller de teste para poder visualizar os resultados dos testes, para ai
				    poder começar os testes. Com um teste criado, é necessário chamá-lo no método index para aparecer
				    o seu resultado relatório, ou seja, todos os testes criados devem ser chamados no método index para
				    conseguir ver algum resultado}.
			  
                        \item \textbf{Essas dificuldades foram reduzidas, ou sanadas, com o uso do \textit{Ignitest}?}
                        
			  \subitem \textit{Com o Ignitest não é preciso mais criar controllers de testes e nem as rotas. Basta criar 
				    a classe de teste e começar os testes. Para ver os resultados do testes também é muito simples, 
				    se resume a um comando!}.
			  
                        \item \textbf{Novas dificuldades apareceram com o uso do \textit{Ignitest}? Se sim, quais?}
                        
			  \subitem \textit{Até então, não}.
			  
                    \end{enumerate}
                \item Quantidade de código
                    \begin{enumerate}
                        \item \textbf{Usando o método padrão, quantas linhas de código foram escritas para o teste da classe escolhida?}
			  
			  \subitem \textit{Com o método padrão o código dos testes possuem em média 8-12 linhas, são muito extensos.
				    Para os 15 casos de testes implementados, foram necessárias 350 linhas em média}.
			  
                        \item \textbf{Com o uso do \textit{Ignitest}, quantas linhas de código foram escritas para o teste da mesma classe?}
			  
			  \subitem \textit{Com o Ignitest o código dos testes se resumiram a 5 linhas em média, o que deixou a classe
				    de testes bastante enxuta, com 175 linhas em média}.
			  
                    \end{enumerate}

                \item Erros
                    \begin{enumerate}
                        \item \textbf{Usando o método padrão, quantos erros ao inserir comandos você cometeu?}
                        
			  \subitem \textit{Foram cometidos 3 erros em relação a adição de dependências na classe de testes e
				    com a definição das rotas.}.
			  
                        \item \textbf{Com o \textit{Ignitest}, quantos erros ao inserir comandos você cometeu?}
			  
			  \subitem \textit{Nenhum erro crítico, apenas erros de digitação do comando}.
                    \end{enumerate}
            \end{itemize}
        
        \item \textit{Feedback}
            \begin{itemize}
                \item \textbf{Como foi sua experiência usando o \textit{Ignitest}?}
		    
		   \subitem \textit{A experiência foi boa. Foi gasto menos tempo com os testes e código do teste está mais
			    enxuto, limpo e elegante}.
		  
                \item \textbf{Durante o uso do \textit{Ignitest}, recebeu alguma mensagem de erro?}
                
		   \subitem \textit{Não}.
		   
                \item \textbf{Durante o uso, o \textit{Ignitest} deixou de funcionar em algum momento?}
                
		   \subitem \textit{Não}.
		   
                \item \textbf{Possui alguma sugestão de melhoria?}
                
		   \subitem \textit{Criar um executável para o Framework, para facilitar a instalação e o uso}.
            \end{itemize}
    \end{itemize}