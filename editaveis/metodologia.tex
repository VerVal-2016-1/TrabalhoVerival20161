Como metodologia de execução do trabalho será realizada uma pesquisa-ação. De acordo com \citeonline{artigo_pesquisa_acao}, 
a pesquisa-ação pode ser caracterizada de várias formas. Para este trabalho será utilizada a pesquisa-ação iterativa e participatória.

Segundo \citeonline{artigo_pesquisa_acao}, a pesquisa-ação tem como objetivo entender uma situação em um contexto prático e melhorar esse contexto por meio de uma ação.
Dessa forma, essa metodologia foi escolhida dado o objetivo de construção de um \textit{framework} de testes para melhoria do \textit{framework} CodeIgniter.

A pesquisa-ação participatória ocorre quando o pesquisador tem uma participação ativa na implementação da ação e na produção das observações acerca do contexto estudado, também participa do compartilhamento de experiências da aplicação da ação. Uma pesquisa-ação iterativa consiste em uma ação dividida em ciclos. \cite{artigo_pesquisa_acao}

A ação a ser proposta é composta de um diagnóstico que é definido por \citeonline{artigo_pesquisa_acao} como uma fase que tem como objetivo principal conhecer a situação atual. Os ciclos de ação definidos para este trabalho tem como base as outras fases definidas por \citeonline{artigo_pesquisa_acao}: Planejamento da ação, Execução da ação e Avaliação da ação.

No diagnóstico serão estudados:
	\begin{itemize}
		\item O \textit{framework} de testes PHPUnit para entender as suas funcionalidades;
		\item O \textit{framework} CodeIgniter para entender como adequá-lo para uso do PHPUnit;
		\item Serão definidas todas as \textit{features} do \textit{framework} a ser criado;
	\end{itemize}

Para cada ciclo serão planejadas as atividades a serem realizadas para implementação do \textit{framework}, após o planejamento essas atividades
serão realizadas. No final de cada ciclo será realizada uma avaliação do \textit{framework} por meio de sua aplicação em um sistema (Sistema Integrado de Gestão Acadêmica). A partir dessa aplicação serão identificadas melhorias para as \textit{features} implementadas e até possíveis novas \textit{features}.