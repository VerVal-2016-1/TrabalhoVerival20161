\chapter{Ciclo 1}
  
  Este capítulo descreve o planejamento realizado e os resultados obtidos com a execução do primeiro ciclo da pesquisa-ação.
  
  \section{Planejamento}
  
      Após a realização do diagnóstico, era necessário um esforço inicial de configuração do \textit{framework} para a 
      adequação do PHPUnit ao CodeIgniter. Além disso, era necessário estabelecer a estrutura do \textit{framework},
      sua arquitetura de funcionamento. Portanto, ficaram definidas como escopo do primeiro ciclo as seguintes atividades:
      
      \begin{itemize}
	
	\item Configurar o CodeIgniter para receber o PHPUnit;
	
	\item Estabelecer arquitetura de funcionamento do \textit{framework}.
	
      \end{itemize}
      
      
      \subsection{Planejamento da avaliação}
      
	  Como este ciclo não agregou muito valor para a atividade de testes dos desenvolvedores do SiGA,
	  embora imprescindível para o \textit{framework}, o mesmo teve uma forma de avaliação diferente
	  dos outros ciclos (não avaliado pelo questionário proposto) por não envolver diretamente os desenvolvedores e se
	  tratar de conteúdo mais técnico inerente à equipe de desenvolvimento do \textit{framework}. Visto isso, ficou proposta
	  a seguinte estrutura de avaliação para as atividades realizadas neste ciclo:
	  
	  \begin{itemize}
	  
	    \item Para a atividade de configuração do CodeIgniter para o uso do PHPUnit:
	  
		\begin{enumerate}
	      
		  \item Testar as configurações propostas em um projeto vazio do CodeIgniter versão 2.x;
		  
		  \item Testar as configurações propostas no SiGA;
		  
		\end{enumerate}
	      
	    \item Para a atividade de estabelecimento da arquitetura de funcionamento do \textit{framework}:
		
		\begin{enumerate}
	
		  \item Validar a arquitetura proposta implementando os comandos \textit{\textbf{init}} e \textit{\textbf{help}};
		      
		      \subitem Se atentar aos seguintes itens indispensáveis:
			\begin{itemize}
			  \item É fácil de se criar um novo comando?
			  \item É possível criar um comando com parâmetros diferentes?
			  \item É possível ter parâmetros opcionais nos comandos?
			  \item É possível tratar cada comando individualmente e executá-los em conjunto?
			\end{itemize}
		  
		  \item Testar o comando \textit{\textbf{init}} em um projeto vazio do CodeIgniter versão 2.x;
		  
		  \item Testar o comando \textit{\textbf{init}} no SiGA;
	
		  \item Testar o comando \textit{\textbf{help}};
		\end{enumerate}
	  
	  \end{itemize}
  
  \section{Execução do ciclo}
  
      Esta seção apresenta detalhes relacionados à execução do ciclo 1.
      
      \subsection{Arquitetura do \textit{Ignitest}}
      
	  
      
      \subsection{Configurações}
	  
	  Para que o PHPUnit funcione no CodeIgniter, descobriu-se que era preciso definir um \textit{hook}\footnotemark 
	  \footnotetext{CodeIgniter Hooks. Disponível em \url{http://www.codeigniter.com/userguide2/general/hooks.html}. Acesso em 20/06/2016.}
	  no projeto para mostrar os resultados do PHPUnit caso o ambiente fosse de teste, pois ao executar o PHPUnit nenhum
	  resultado era disponibizado na tela. Foi preciso seguir os seguintes passos para resolver esse problema:
	  
	  \begin{itemize}
	  
	   \item Habilitar o uso de \textit{hooks} no CodeIgniter no arquivo '\textbf{application/config/config.php}':
	      
	      \begin{verbatim}
		$config['enable_hooks'] = TRUE;
	      \end{verbatim}
	      
	   \vfill
	   \pagebreak
	   \item Criar o seguinte \textit{array} no arquivo '\textbf{application/config/hooks.php}':
	  
	      \begin{lstlisting}
		$hook['display_override'] = array(
		    'class' => 'DisplayHook',
		    'function' => 'captureOutput',
		    'filename' => 'DisplayHook.php',
		    'filepath' => 'hooks'
		);
	      \end{lstlisting}
	   
	   \item Criar a seguinte classe no diretório '\textbf{/application/hooks/}':
	  
	  \begin{lstlisting}
	      class DisplayHook {
	          public function captureOutput() {

	              $this->CI =& get_instance();
			  
	              $output = $this->CI->output->get_output();

	              if (ENVIRONMENT != 'testing') {
	                  echo $output;
	              }
	          }
	      }
	  \end{lstlisting}
	  
	  \end{itemize}
	  
	  Além dos passos acima, são necessárias as configurações essenciais do PHPUnit, que são descritas abaixo:
	  
	  \begin{itemize}
	    \item Criar o arquivo de configuração do PHPUnit '\textbf{\textit{phpunit.xml}}' 
		  no diretório '\textbf{application/tests/}';
	    
	    \item Criar o arquivo de inicialização para os testes '\textbf{\textit{bootstrap.php}}'
		  no diretório '\textbf{application/tests/}';
	      
		\subitem Este arquivo é executado antes dos testes. Para o uso no CodeIgniter, esse arquivo deve ser
			 adaptado para conter o mesmo conteúdo do arquivo de inicialização do CodeIgniter '\textbf{index.php}', 
			 para que os recursos do CodeIgniter estejam disponíveis durante os testes.
	  \end{itemize}
	  
	  Durante a execução do ciclo, percebeu-se que era necessária bastante configuração para que o PHPUnit rodasse no
	  CodeIgniter e, portanto, parte dessa configuração, as configurações essenciais do PHPUnit, foi abstraída para
	  dentro do \textit{framework} no comando \textit{\textbf{init}}, para diminuir a carga de configuração
	  para o desenvolvedor.
  
  \section{Resultados obtidos}
    
      Este ciclo produziu os seguintes resultados:
      
      \begin{itemize}

	\item Configuração do PHPUnit ao CodeIgniter estabelecida;
	
	\item Arquitetura do \textit{framework} estabelecida;
	
	\item Comando \textit{\textbf{init}}, para automatizar parte da configuração.

      \end{itemize}

  \section{Avaliação dos resultados}
  
      \subsection{Melhorias identificadas}
    

\chapter{Ciclo 2}

  Este capítulo descreve o planejamento realizado e os resultados obtidos com a execução do segundo ciclo da pesquisa-ação.
  
  \section{Planejamento}
  
      Para compor o escopo do segundo ciclo ficaram alocadas as seguintes funcionalidades, com suas respectivas características:
      
     \begin{itemize}
      \item \textbf{Criação da classe base para testes unitários;}
	\begin{itemize}
	  \item Reconhecimento da classe que está sendo testada por convenção de nomenclatura;
	\end{itemize}

      \item \textbf{Classe base para testes de integração.}
	\begin{itemize}
	  \item Reconhecimento da classe que está sendo testada por convenção de nomenclatura;
	  \item Esquema de criação e destruição do banco de testes;
	  \item Definição de dados iniciais no banco de testes;
	\end{itemize}
      \end{itemize}

  \section{Execução do ciclo}
      
      Sincronização dos banco de dados.
  
  \section{Resultados obtidos}
  
  
  \section{Avaliação dos resultados}
  
    \subsection{Melhorias identificadas}
    
\chapter{Ciclo 3...n}
  
  O tempo para a realização do trabalho não permitiu a execução de mais de um
  ciclo,embora o planjemento inicial tenha previsto mais de um ciclo. Todavia, sabe-se que o escopo dos próximos
  ciclos seria estabilizar as funcionalidades propostas do \textit{framework} com base nas melhorias identificadas
  no final de cada ciclo.
  
  A quantidade de ciclos foi estabelecida como indefinida, pois seriam executados
  ciclos até que o \textit{framework} se tornasse estável.
  